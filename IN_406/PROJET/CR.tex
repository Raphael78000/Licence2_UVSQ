\documentclass{article}
\usepackage{graphicx}
\usepackage[utf8]{inputenc}

\title{PROJET - IN406 - THEORIE DES LANGAGES}
\author{LAZZARI-ARMOUR Raphael et POULIQUEN Chloé}
\date{Mai 2021}

\begin{document}

\maketitle

\section{Question 1}
\textit{\underline{Question:} Lire une chaîne de caractère contenant une expression arithmétique et la transformer en une liste de tokens.} 


\section{Question 2}
\textit{\underline{Question:} Donner un automate à pile reconnaissant le langage dont les mots sont les expressions
booléennes.}

\section{Question 3}
\textit{\underline{Question:} Écrire une fonction en langage C qui teste si une liste de token appartient au langage ou non.}

\section{Question 4}
\textit{\underline{Question:} À partir de la liste de tokens et en utilisant l’automate à pile, créer l’arbre représentant l’expression booléenne. Vous pouvez utiliser la fonction qui teste si la liste des tokens appartient au langage en la modifiant.}

\section{Question 5}
\textit{\underline{Question:} Calculer la valeur de l’expression arithmétique et afficher le résultat}

\end{document}
